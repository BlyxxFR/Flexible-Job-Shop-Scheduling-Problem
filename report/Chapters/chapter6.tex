\section{Évaluation de la qualité des solutions renvoyées par l'approche génétique}

\subsection{Méthode d'évaluation de la qualité}



\subsection{Résultats obtenus}

Nous avons décidé de confronter notre algorithme aux résultats connus pour le \textit{Brandimarte_Data}. Dans les tableaux suivants, LB correspond aux bornes inférieures connues et UB aux bornes supérieures connues de l'optimum. $n$ correspond au nombre de job et $m$ au nombre de machines.

\subsubsection{Résultats de notre algorithme pour une population de 200 individus et une génération maximale de 500}

\begin{table}[!h]
    \renewcommand{\arraystretch}{1.5}
    \centering
    \begin{tabular}{p{\textwidth/7} c c c c c}
        Instance & $n \times m$ & LB & UB & Temps moyen & Temps d'exécution moyen \\
         \hline
        Mk01 & $10 \times 6$ & \textbf{40} & \textbf{40} & & \\
         \hline
        Mk02 & $10 \times 6$ & \textbf{26} & \textbf{26} & & \\
         \hline
        Mk03 & $15 \times 8$ & \textbf{204} & \textbf{204} & & \\
         \hline
        Mk04 & $15 \times 8$ & \textbf{60} & \textbf{60} & & \\
         \hline
        Mk05 & $15 \times 4$ & \textbf{172} & \textbf{172} & & \\
         \hline
        Mk06 & $10 \times 15$ & \textbf{57} & \textbf{57} & & \\
         \hline
        Mk07 & $20 \times 5$ & \textbf{139} & \textbf{139} & & \\
         \hline
        Mk08 & $20 \times 10$ & \textbf{523} & \textbf{523} & & \\
         \hline
        Mk09 & $20 \times 10$ & \textbf{307} & \textbf{307} & & \\
         \hline
        Mk10 & $20 \times 15$ & 189 & 193 \\
         \hline 
    \end{tabular}
\end{table}

\subsubsection{Résultats de notre algorithme pour une population de 500 individus et une génération maximale de 1000}

\begin{table}[!h]
    \renewcommand{\arraystretch}{1.5}
    \centering
    \begin{tabular}{p{\textwidth/7} c c c c c}
        Instance & $n \times m$ & LB & UB & Temps moyen & Temps d'exécution moyen \\
         \hline
        Mk01 & $10 \times 6$ & \textbf{40} & \textbf{40} & & \\
         \hline
        Mk02 & $10 \times 6$ & \textbf{26} & \textbf{26} & & \\
         \hline
        Mk03 & $15 \times 8$ & \textbf{204} & \textbf{204} & & \\
         \hline
        Mk04 & $15 \times 8$ & \textbf{60} & \textbf{60} & & \\
         \hline
        Mk05 & $15 \times 4$ & \textbf{172} & \textbf{172} & & \\
         \hline
        Mk06 & $10 \times 15$ & \textbf{57} & \textbf{57} & & \\
         \hline
        Mk07 & $20 \times 5$ & \textbf{139} & \textbf{139} & & \\
         \hline
        Mk08 & $20 \times 10$ & \textbf{523} & \textbf{523} & & \\
         \hline
        Mk09 & $20 \times 10$ & \textbf{307} & \textbf{307} & & \\
         \hline
        Mk10 & $20 \times 15$ & 189 & 193 \\
         \hline 
    \end{tabular}
\end{table}