\section{Résolution à l'aide d'une heuristique}

\subsection{Objectifs de la méthode}

Cette méthode permet de trouver une borne supérieure de l'optimum très rapidement. Elle consiste à appliquer une heuristique au moment de choisir quelles opérations les machines vont traiter.

\subsection{Algorithme du scheduler}

\subsubsection{Instanciation de la classe, paramètres de la méthode et initialisation}

\lstinputlisting[firstline=4, lastline=10]{./app/scheduler.py}

Le constructeur de la classe prend trois paramètres, la liste des machines, le nombre d'opérations en parallèle maximal et la liste des jobs construite avec le parseur. La variable \textit{original\_stdout} permet de supprimer la sortie texte et la rétablir au besoin.

\begin{lstlisting}
s = Scheduler(machines_list, number_max_operations, jobs_list)
\end{lstlisting}

Pour démarrer le scheduler, il suffit d'appeler la méthode \textit{run} en passant en paramètre l'heuristique considérée et optionnelement mettre \textit{verbose} à \textit{True} ou \textit{False} en fonction de si l'on veut un affichage ou non.
\begin{lstlisting}
s.run(Heuristics.select_first_operation)
\end{lstlisting}

Lors de cet appel, la méthode \textit{run} va commencer par supprimer la sortie standard si indiqué et initialiser la variable \textit{current\_step} à $0$. Cette variable représente l'instant $t$ auquel le système est. 

\lstinputlisting[firstline=12, lastline=17]{./app/scheduler.py}

\newpage

\subsubsection{Déroulement de l'algorithme}

\lstinputlisting[firstline=20, lastline=38]{./app/scheduler.py}

La variable \textit{best\_candidates} correspond au retour de l'heuristique considérée. Il s'agit d'un dictionnaire associant aux identifiants des machines la liste des opérations (ainsi que l'activité auxquelles elles sont rattachées) qu'elles devraient traiter à l'instant $t$.

La boucle \textit{for} parcourant \textit{best\_candidates.items()} permet d'affecter l'opération sur la bonne machine.

La boucle \textit{for} parcourant \textit{self.\_\_machines} permet de simuler le travail des machines pendant une unité de temps.

\subsubsection{Fin de la méthode \textit{run}}

\lstinputlisting[firstline=40, lastline=47]{./app/scheduler.py}

On affiche la durée totale que le planning prend et on réactive \textit{std\_out} si nécessaire. 

\subsection{Choisir l'opération la plus courte dans une activité comme heuristique}

\lstinputlisting[firstline=2, lastline=29]{./app/heuristics.py}