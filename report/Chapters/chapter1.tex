\section{Introduction}

\subsection{Présentation du projet}

Dans ce projet, nous considérons un problème d'ordonnancement appelé "Flexible Job Shop". Nous disposons d'un ensemble de $n$ travaux (jobs) devant être exécutés sur $m$ machines. Chaque job se décompose en une liste d'activités qui doivent être réalisées dans l'ordre. Une activité est décrite par un ensemble d'opération de durée et de machine différente et il faut choisir l'opération qui minimise la durée totale que nécessite l'ensemble des jobs pour être terminé.

On supposera que les machines ne peuvent réaliser qu'une opération à la fois bien que la solution que nous proposons dans le cas d'une recherche à l'aide d'une heuristique, les machines peuvent supporter plusieurs opérations en simultané. Dans le cas de l'approche génétique, ce nombre est fixé à une seule opération en simultané.

\subsection{Structure des jeux de données}



\subsection{Structure de notre projet}

\subsubsection{Classe représentant les jobs}

\subsubsection{Classe représentant les activités}

\subsubsection{Classe représentant les opérations}

\subsubsection{Classe représentant les machines}
